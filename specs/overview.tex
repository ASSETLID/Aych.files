Even in languages with strong static typing, style-checkers can be
very useful: some coding styles are known to ease the hidden presence
of bugs; other coding styles can be inefficient or raise memory
issues. Also, having different coding-styles in a project can make
code-review more difficult, disturbing the focus of reviewers towards
minor style issues instead of looking for algorithmic issues.

A style-checker can solve many of these problems by providing an
automatic way to check for coding styles that should be avoided in a
project. For that, the style-checker should be fast, to provide
feedback almost immediately, configurable, to allow project leaders to
express their preferences, and extensible, to allow project developers
to design new analysis specific to their needs.

The main design idea behind \ocplint{} is to provide a framework for
checking OCaml projects for coding errors, instead of just another
monolithic tool. For that, we have tried to make it easy to
extend \ocplint{}, either using \emph{semantic patches} (patterns of
code described in a patch-like syntax on OCaml code) or using dynamic
plugins (user code linked at runtime). We also tried to make it as
configurable as possible: each plugin and each analysis can be enabled
or disabled by a simple project configuration file, and analysis
settings can be modified in the same configuration file. We wrote a
set of plugins and analyses, both to be able to use the tool for our
own purposes, and to provide examples for developers of how to extend
the tool.

Configuration files in \ocplint{} are managed by the {\tt
ocplib-config}~\cite{ocplib-config} library. This library provides
simple functions to define options, that can be manipulated as simply
as references in the program, while being loaded and saved
automatically in configuration files. The library also automatically
generates command-line arguments to modify the options. \ocplint{} can
use both user- and directory- specific configuration files.

We already implemented a small set of plugins working on different
kinds of inputs: {\sf text} plugin (length of lines, extra spaces,
non-ANSII characters, etc.), {\sf indent} plugin (correct use of {\sf
ocp-indent}), {\sf tokens} plugin (non-ASCII characters in comments,
in idents, etc.), {\sf parsetree} plugin (constructor with tuple
arguments, local aliases, identifiers lengths, semantic patches,
etc.), {\sf sempatch} plugin (detection of patterns provided by
semantic patches), {\sf typedtree} plugin (non-qualified external
idents), {\sf parsing} plugin (extra-parentheses, use of parentheses instead of
 {\sf begin..end}, etc.).
 
Warnings emitted by these analyses are stored in a
project-database. The database is then used when the tool is
restarted, to avoid checking files again if they have not been
modified, and to allow other tools to display the results ({\sf
ocp-index} or {\sf Merlin} for example, or on a web interface).

In this document, we will present \ocplint{}, a style-checker for OCaml
projects, that aims at satisfying all these needs. An overview of
\ocplint{} is given in section~\ref{sec:overview}. Of course,
\ocplint{} builds on our experience learnt from using other
style-checkers for OCaml, as shown in section~\ref{sec:related}. It is
easy to use and configure, as depicted in appendix~\ref{sec:usage}, and
can be extended using both \emph{semantic patches}, in
appendix~\ref{sec:sempatch}, and dynamic plugins using a simple API,
described in appendix~\ref{sec:extensibility}.
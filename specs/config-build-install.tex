
\def\ocpautoconf{{\tt ocp-autoconf} }
\def\ocpbuild{{\tt ocp-build} }

\subsection{Dependencies}
There are a few dependencies for \ocplint: menhir, ocp-indent and ocp-build. You
can install them via OPAM, the OCaml PAckage Manager:

{\tt \$ opam install menhir ocp-build ocp-indent}

\subsection{Configuration}
First, we need to configure {\tt typerex-lint}. To do so, we use
\ocpautoconf~\cite{ocp-autoconf}, a simple tool to manage standard project
files for OCaml projects. It can manage:
\begin{itemize}
\item {\em ./configure}: basic configure.ac files (autoconf) to detect OCaml and
  its libraries, and set variables to be used in Makefiles and `ocp-build`
  files.
\item {\em opam}: generate a standard opam file for your project, and a script to
  upload new versions of your project to Github
\item {\em Travis files}: standard Travis files to test pull-requests on Github
\end{itemize}

\paragraph{Basic Usage} \ocpautoconf will create a `configure' file and a
'autoconf/' directory at the root of your project, so you should make sure such
files do not already exist. The `configure` file is just a script to call the
real `configure` script inside the `autoconf/` directory. 

{\tt \$ ocp-autoconf \--\--save-template}\\

This will create two configuration files for \ocpautoconf: 
\begin{itemize}
\item {\em ocp-autoconf.config}: a file containing options to describe your
  project 
\item {\em ocp-autoconf.ac}: a file that you can use to insert `autoconf`
  instructions inside the `autoconf/configure.ac` file that will be generated.
\end{itemize}

Then we can start the configure script:

{\tt \$ ./configure}


\subsection{Build}
\ocpbuild is a build system for OCaml application, based on simple descriptions
of packages. \ocpbuild combines the descriptions of packages, and optimize the
parallel compilation of files depending on the number of cores and the
automatically-infered dependencies between source files.

For building the project you can either use directly \ocpbuild or the Makefile
which will just call \ocpbuild:

{\tt \$ make} 

\subsection{Installation}
You can install \ocplint{} via OPAM:

{\tt \$ opam install ocp-lint}

~\\
Or pin the development package:

{\tt \$ opam pin add ocp-lint \$OCPLINTSOURCES}

{\tt \$ opam install ocp-lint}

~\\
You can install \ocplint{} directly from the sources:

{\tt \$ make install}

\subsection{Testsuite}
Each plugin can define a set of tests. To make all the test easily we use
\ocpbuild which will find the .ocp file containing the tests.
To define a set of test, create a directory in a plugin source:

{\tt \$ mkdir \$OCPLINTPATH/plugins/MYPLUGIN/tests}

~\\
Then create a .ocp file to describe the tests:

\begin{lstlisting}
begin test "test-ocp-lint-MYPLUGIN"
  test_byte = false
  requires = [ "ocp-lint-testsuite" ]
  test_args = [
    "%{ocp-lint_FULL_DST_DIR}%/ocp-lint.asm" "%{sources}%"
  ]
end
\end{lstlisting}

~\\

{\bf TODO cago: michael, il faudrait que tu décrives le process complet pour
  ajouter un test avec les histoires de .result}

~\\
Finally you can use ocp-build to run the tests anywhere in the source tree:

{\tt \$ ocp-build tests}

~\\
Or in the source directory use the Makefile rule:

{\tt \$ make test}
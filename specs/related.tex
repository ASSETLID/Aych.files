\ocplint{} is not the only tool that can be used to improve the quality of
the code of an OCaml project. In this section, we compare \ocplint{} with
three other tools that can be used for this purpose.

{\em Mascot}~\cite{mascot} was probably the most exhaustive
style-checker for OCaml. It provided many checks in various
categories: code, documentation, interface, metrics, and
typography. However, it is not maintained anymore, and hard to extend,
especially as analyses are heavily based on using Camlp4 syntax trees.

{\tt ocamllint}~\cite{ocamllint} is a style-checker that runs as {\sf
ppx}~\cite{ppx-blog} while compiling the project.  Thus, it
requires minimum effort to be used on an OCaml project.  However, the
number of analyses is currently very limited, and they can only be
applied on the AST, whereas \ocplint{} can work also on text files,
and on typedtrees.

{\em Dead code analyzer}~\cite{DeadCodeAnalyzer} tries to detect
useless patterns in an OCaml project.  For example, it detects never
used values, types fields and constructors (that can thus be removed
as dead code), and optional labels either always or never used. The
tool assumes that interface files ({\tt .mli}) are compiled with the
{\tt -keep-locs} and source files ({\tt .ml}) with {\tt
-bin-annot}. The analysis can be quite expensive, but the tool is a
good complement to \ocplint{}, and could be added as a plugin to
benefit from its database and project management.

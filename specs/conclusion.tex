When designing \ocplint{}, our goal was to use it on our own Github
projects, to check pull-requests both from the project developers and
from external contributors. We plan to apply it soon to all our
projects, once most of the analyses we need are implemented.

Although we currently use a non-optimized approach in the
implementation of the analyses (different checks are often done in
different analyses, while they could be done in the same iteration on
the AST), performance is good enough for its purpose. For example,
running \emph{sequentially} all the current 30 analysis of our 6 plugins, we
get the following performances:\\
{\small\noindent
\begin{tabular}{|l|l|l|l|l|}
  \hline
  Project & Files & LOC & Warnings & Time \\
  \hline
  ocp-index & 12 & 4333 & 36 & 0.14s \\
  \hline
  ocp-indent & 12 & 5763 & 44 & 0.32s \\
  \hline
  stdlib & 35 & 12957 & 80 & 2.18s \\
  \hline
  opam & 64 & 26906 & 362 & 3.09s \\
  \hline
  flow & 119 & 47833 & 563 & 13.25s \\
  \hline
  hack & 386 & 73715 & 1213 & 33.57s \\
  \hline
\end{tabular}}\\

We also took internationnalization into account in the design: the
message associated with each warning is a simple string, that will be
customized in the future for different languages.

The project sources are hosted on Github, and an OPAM package should
be available soon.

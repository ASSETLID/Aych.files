\subsection{Using Occasionally}
\paragraph{Without the database}~\\

To start to lint your project without the database feature, just run \ocp-lint{}
which will output the warnings in your terminal by default:

  {\tt \$ cd \$PROJECT}

  {\tt \$ ocp-lint}

Without the database feature, nothing is cached, so every time you will run
\ocp-lint{}, it will recompute and re-lint every file on your project, even when
it is no needed.

\paragraph{With the cache system (database)}~\\

First go to your project directory and initialize the linter which will create a
default configuration file and an {\tt \_olint} directory to store the the
results for the cache system.

  {\tt \$ cd \$PROJECT}

  {\tt \$ ocp-lint \--\--init \# create \_olint directory and .ocplint
    configuration file}

~\\  
Then, edit your .ocplint file or just run the default behavior of the linter.

  {\tt \$ ocp-lint \# default directory is .} 

  {\tt \$ ocp-lint \--\--path src/ }

You will get some entries in your database showing the different warning of your
project. 

\begin{lstlisting}[language=bash,basicstyle=\tt\small]
$ ocp-lint --path $SOURCEDIR --disable-plugin-typedtree
Summary:
* 11 files were linted
* 40 warnings were emitted:
 * 2 "interface_missing" number 1
 * 2 "code_length" number 1
 * 4 "ocp_indent" number 1
File "lint_input.ml", line 1:
  "ocp_indent" number 1:
   File lint_input.ml' is not indented correctly.
File "lint_actions.ml", line 1:
  "code_length" number 1:
  This line is too long ('82'): it should be at most of size '80'.
File "main.ml", line 1:
  "interface_missing" number 1:
  Missing interface for main.ml'.
\end{lstlisting}              

After that, you can edit your source files to fix the issues raised by the
linter.

To force the linter to lint a file, even when it is no needed, we can use the
{\tt \--\--no-db-cache} option in the command line. The linter will force the
lint of your files and will not cache the result in the database.

You can run \ocplint{} as often as you want, it will update your
database only if you change your current configuration or if a file has changed.


\subsection{Using During the Development}

The initializing process is the same as explain in the previous section.

\ocplint{} can be integrated to your editor as {\tt ocp-index} or {\tt
  merlin}. You can then have the linter output directly in your editor and then
go to the warning line directly. With this feature, every warning could be fixed
more quickly and during the development process.

Some hook can also be add to start the linter after each compilation of your
project or even each time you save a file. In that case, only this file can be
linted to see the output in your editor, it will be more easy to fix the
warnings.

\subsection{Using in a Git-development process}
Another use of \ocplint{} is to integrate it during the development process. For
example, we can add some hook with {\tt git} which will check the project each
time a git command will be executed.

\subsubsection*{As a Pre-commit Hook}
One of these hooks is to check the project before committing. To do so, we can
add a {\em pre-commit hook} which will start \ocplint{} before each commit. We
already propose the script below that you can add to your git configuration:

\begin{lstlisting}[language=Bash]
#!/bin/sh

LINT=ocp-lint

$LINT --warn-error tools > /dev/null

if [ "$?" = 0 ]; then
    exit 0
else
    echo "\n/!\\ Please fix the warnings before commiting./!\\"
    exit 1
fi
\end{lstlisting}

You can copy this script in your {\tt \$PROJECT/.git/hook} directory to enable
the pre-commit hook. The linter will warn you about your code before each
commit.

\subsubsection*{As a Pre-push Hook}
In a similar way, you can add a pre-push hook which will warn you about the
warnings in your code before every {\tt git push} command. Until every warnings
are fixed, there will be any chance to push the code to your git repository.

\subsection{Using in a CI Process}
Using \ocplint{} with a CI tool like travis, you can add the hook to start the
linter after the compilation process. Then, the CI tool can automatically add
comments to your github pull request with all the warnings or directly add
comments to the corresponding lines. Developers can then discuss about the
warnings and decided the needed to fix them or not to accept the pull request.

\ocplint{} can be more strict with the {\tt --warn-error} option. In that case,
the linter will return an exit code different from 0 and the CI tool will fail
until all warnings are fixed.

